\documentclass{ltxdockit}[2011/03/25]
\usepackage{btxdockit}
\usepackage[main=english,french,latin]{babel}

\usepackage{fontspec}
\usepackage[mono=false]{libertine}
\usepackage{microtype}
\usepackage[strict]{csquotes}
\setmonofont[Scale=MatchLowercase]{DejaVu Sans Mono}
\usepackage{shortvrb}
\usepackage{minted}
\usepackage{pifont}

% Usefull commands
\newcommand{\biblatex}{biblatex\xspace}
\pretocmd{\bibfield}{\sloppy}{}{}
\pretocmd{\bibtype}{\sloppy}{}{}
\newcommand{\bibkey}[1]{\texttt #1}
% Meta-datas
\titlepage{%
	title={Source division with biblatex},
	email={maieul <at> maieul <dot> net},
	author={Maïeul Rouquette},
	subtitle={},
	revision={1.0.0-beta},
	date={7/12/2013},
	url={https://github.com/maieul/biblatex-source-division}}
	

\begin{document}

\printtitlepage

\tableofcontents
\section{Introduction}
\subsection{Goals}
The \biblatex package allows to refers to a precise page number when citing a references:
\begin{minted}{latex}
\cite[23]{key}
\end{minted}

Means \enquote{cite the entry {\bibkey key}, and precise this we are refering to the p.~23 of this entry.}

However, historian or philologist can want to precise an other information: the source division, which for an old texte is independant of the edition. Mostly, this source division is something like: book, chapter, section, but it can have an other scheme. The source division is printed after the book name, but before the publication information (translator, address, publisher). The book division doesn't prevent to print the page number. 

For example,  citing the work of Augustine \emph{De Doctrina Christiana} in the book~II, chapter~\textsc{viii} section~13 in French translation of the \emph{\selectlanguage{french}Bibliothèque Augustinienne} and in reference edition of the \emph{\selectlanguage{latin}Corpus Christianorum Series Latina} will produce:

% mettre reference
% mettre reference

\subsection{Credits}

This package was created for Maïeul Rouquette's phd dissertation\footnote{\url{http://apocryphes.hypothese.org}} in 2013. It is licenced on the \emph{\LaTeX\ Project Public Licence}\footnote{\url{http://latex-project.org/lppl/lppl-1-3c.html}.}. 
Its code is freely inspired of a contribution of Andrey Boruvka\footnote{\url{http://tex.stackexchange.com/q/95110/}.}.

All issues can be submitted, in French or English, in the GitHub issues page\footnote{\url{https://github.com/maieul/biblatex-manuscripts-philology/issues}.}.


\end{document}